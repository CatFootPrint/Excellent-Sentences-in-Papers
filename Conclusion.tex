% !Mode:: "TeX:UTF-8" %
\documentclass[onecolumn,conference]{IEEEtran}
\usepackage{mathrsfs}
\usepackage{amssymb}
\usepackage{amsmath}
\usepackage{amssymb}
\usepackage{amsthm}
\usepackage{multicol}
\usepackage{comment}
\usepackage{graphics}
\usepackage{picins}
\usepackage{graphicx,subfigure}
\newtheorem{Theo}{Theorem}
\newtheorem{Lemm}{Lemma}
\usepackage{multirow} 
\usepackage[xetex,colorlinks,pagebackref,linkcolor=blue,citecolor=blue,urlcolor=blue]{hyperref}
\usepackage{array}
\usepackage{stfloats}
\usepackage{tabularx}
\usepackage{float}
\usepackage[ruled]{algorithm2e}
\usepackage{siunitx}
\usepackage{setspace}
\begin{document}
%\begin{spacing}{3.0}%%?????double-space
\title{Experimental Results}
\author{\IEEEauthorblockN{Kezhong Zhang}\today}
\maketitle
\begin{enumerate} 
\item Moreover, compared with other MLP training methods, the training of H-ELM is much faster and achieves higher learning accuracy. \cite{Tang2015}
\item Our studies have revealed that sparse coding can provide a promising avenue for content fingerprinting with the following merits. \cite{Yue2015}
\item In this letter, we proposed a novel anchor-based local learning method for SISR by joint learning of the feature space partition and local regressors. \cite{Zhang2016}
\end{enumerate}
\bibliographystyle{IEEEtran}
\bibliography{reference}
\end{document}