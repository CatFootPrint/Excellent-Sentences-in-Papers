% !Mode:: "TeX:UTF-8" %
\documentclass[onecolumn,conference]{IEEEtran}
\usepackage{mathrsfs}
\usepackage{amssymb}
\usepackage{amsmath}
\usepackage{amssymb}
\usepackage{amsthm}
\usepackage{multicol}
\usepackage{comment}
\usepackage{graphics}
\usepackage{picins}
\usepackage{graphicx,subfigure}
\newtheorem{Theo}{Theorem}
\newtheorem{Lemm}{Lemma}
\usepackage{multirow} 
\usepackage[xetex,colorlinks,pagebackref,linkcolor=blue,citecolor=blue,urlcolor=blue]{hyperref}
\usepackage{array}
\usepackage{stfloats}
\usepackage{tabularx}
\usepackage{float}
\usepackage[ruled]{algorithm2e}
\usepackage{siunitx}
\usepackage{setspace}
\begin{document}
%\begin{spacing}{3.0}%%?????double-space
\title{Experimental Results}
\author{\IEEEauthorblockN{Kezhong Zhang}\today}
\maketitle
\begin{enumerate} 
\item In all the simulations below, the testing hardware and software conditions are listed as follows: Laptop, Intel-i7 2.4G CPU, 16G DDR3 RAM, Windows 7, MATLAB R2013b. \cite{Tang2015}
\item Fig. 2(a) and (b) \textbf{shows} the learning accuracies of H-ELM and ELM in the L subspace, where the parameter C is prefixed. It can be seen that H-ELM seems to follow a similar convergence property of ELM but with higher testing accuracy, and the performances tend to be quite stable in a wide range of L. \cite{Tang2015}
\item One can easily observe that, the proposed H-ELM has a remarkable \textbf{improvement against} the ELM with most of the testing data sets, in terms of learning accuracy.
 \cite{Tang2015}
\item Furthermore, from Table V, we can see that the training time of the H-ELM-based method is much \textbf{less than that of} the DL-based SDA one, even with higher EER. \cite{Tang2015}
\item Moreover, the testing scenes \textbf{are shown in} Fig. 7, where Rows 1 and 4 are the results obtained from H-ELM, and the other rows are the results from CT and SDA. \cite{Tang2015}
\item As can be seen in Rows 1–3 (David Indoor data set), the tracking locations of H-ELM is more accurate than those of CT and SDA. \cite{Tang2015}
\item Besides, by the advantages of fast training of H-ELM framework, H-ELM
achieves real-time tracking (over 15 frames/s), while the result for SDA is only 1–2 frames/s. \cite{Tang2015}
\item The performance of the proposed algorithm was \textbf{evaluated} by content identification experiments under the following parameter settings. \cite{Yue2015}
\item Fig. 2(b) displays the ROC curves corresponding to the ordinary and incoherent dictionaries, which clearly shows the superiority of the incoherent one. \cite{Yue2015}
\end{enumerate}
\bibliographystyle{IEEEtran}
\bibliography{reference}
\end{document}