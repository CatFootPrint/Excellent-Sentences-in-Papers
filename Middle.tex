% !Mode:: "TeX:UTF-8" %
\documentclass[onecolumn,conference]{IEEEtran}
\usepackage{mathrsfs}
\usepackage{amssymb}
\usepackage{amsmath}
\usepackage{amssymb}
\usepackage{amsthm}
\usepackage{multicol}
\usepackage{comment}
\usepackage{graphics}
\usepackage{picins}
\usepackage{graphicx,subfigure}
\newtheorem{Theo}{Theorem}
\newtheorem{Lemm}{Lemma}
\usepackage{multirow} 
\usepackage[xetex,colorlinks,pagebackref,linkcolor=blue,citecolor=blue,urlcolor=blue]{hyperref}
\usepackage{array}
\usepackage{stfloats}
\usepackage{tabularx}
\usepackage{float}
\usepackage[ruled]{algorithm2e}
\usepackage{siunitx}
\usepackage{setspace}
\begin{document}
%\begin{spacing}{3.0}%%?????double-space
\title{Introduction}
\author{\IEEEauthorblockN{Kezhong Zhang}\today}
\maketitle
\begin{enumerate}
\item Given an over-complete dictionary $\mathbf{D}=\begin{bmatrix}
\mathbf{d}_1&\cdots\mathbf{d}_K
\end{bmatrix}\in \mathbb{R}^{5\times K}$ with $K$ atoms, sparse coding seeks to representing a feature vector $\mathbf{y}_i$ using at most $T_0$ atoms:
\begin{equation}
\min_{\mathbf{x}_i\in \mathbb{R}^K}
\begin{Vmatrix}
\mathbf{y}_i-\mathbf{D}\mathbf{x}_i
\end{Vmatrix}_2, \qquad \text{s.t.}\quad
\begin{Vmatrix}
\mathbf{x}_i
\end{Vmatrix}_0\leqslant T_0\cite{Yue2015}
\end{equation}
\item From a robustness point of view, sparse vectors are desired to be stable against content-preserving manipulations. \cite{Yue2015}
\item K-SVD [27] is one of the most effective algorithms for dictionary learning, in which approximation error is minimized by alternatively updating $\mathbf{X}$ and $\mathbf{D}$. However, it does not take the mutual coherence constraint into account. \cite{Yue2015}
\item Denote the training set by $\mathbf{V}=\left[\mathbf{v}_1,\cdots,\mathbf{v}_S\right]$, then the objective of dictionary learning can be expressed as
\begin{equation}
\begin{aligned}
&\text{minimize}\quad
\begin{Vmatrix}
\mathbf{V-DX}
\end{Vmatrix}_2\\
&\text{subject to} \quad \mu\left(\mathbf{D}\right)\leqslant\mu_0;\quad \begin{Vmatrix}
\mathbf{x}_i
\end{Vmatrix}\leqslant T_0,1\leqslant i\leqslant S
\end{aligned}\cite{Yue2015}
\end{equation}
\item Algorithm 1 \textbf{outlines} the procedures of dictionary learning. \cite{Yue2015}
\item \textbf{In contrast to} RVFL theories for semi-randomness, ELM theories show that 
\begin{enumerate}
\item Generally speaking, all the hidden node parameters can be randomly generated as long as the activation function is nonlinear piecewise continuous; 
\item all the hidden nodes can be not only independent from training samples but also \textbf{independent from} each other; 
\item ELM theories are valid for but not limited to sigmoid networks, RBF networks, threshold networks, trigonometric networks, fuzzy inference systems, fully complex neural networks, high-order networks, ridge polynomial networks, wavelet networks, Fourier series, and biological neurons whose modeling/shapes may be unknown, \emph{etc.} [9], [10], [13].
\end{enumerate}
 \cite{Huang2015}
\end{enumerate}
\bibliographystyle{IEEEtran}
%\IEEEtriggeratref{13}
\bibliography{reference}
%\end{spacing}
\end{document}
